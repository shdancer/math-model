\documentclass{cumcmthesis}
\usepackage[framemethod=TikZ]{mdframed}
\usepackage{url}   % 网页链接
\usepackage{subcaption} % 子标题
\title{液滴问题}\tihao{A}
\baominghao{4321}
\schoolname{你的大学}
\membera{成员A}
\memberb{成员B}
\memberc{成员C}
\supervisor{指导老师}
\yearinput{2017}
\monthinput{08}
\dayinput{22}
\begin{document}
\maketitle

\begin{abstract}    水滴铺展问题是一个涉及到流体力学,表面现象等的综合性问题。人类
    对水滴在固体基板上铺张与稳定的研究从只能研究稳定态的杨氏方程开始,从表
    面吉布斯函数和表面张力出发,一直探索到拉普拉斯方程,直到近年来中国科学
    技术大学的最新研究成果杨氏方程的微观力学解释......小小的液滴之中蕴含
    了无数深刻的物理学知识。而如若液滴非稳定,其运动状态又只能用纳维尔-斯
    托克斯方程这样经典但又深奥难以解决的方程来描述。 
    我们学习研究了前人的成果,在这基础上对几种不同初始情况下的液滴建立
    了不同的模型来拟合它们形态各异的铺展现象,对比了不同模型之间结果的异
    同,并且讨论了他们的适用范围和模型建立的合理性。 
    在求解数学模型时,我们又运用了不同的数值计算方法去来求解一阶常微分
    方程和二阶偏微分方程等难以求得解析解的模型,还使用Matlab、COMSOL等软
    件对数值解进行可视化,且在Matlab中设计了合理的算法对第一问的模型进行综
    合求解。 
    \cite{mckean1970nagumo}
    \cite{rossler1979equation}
    \cite{rossler1979equation}
    针对问题一,我们论述了重力对于液滴形态影响极小,可以忽略不计,因此
    为了易于确定液滴铺展的形变过程,我们提出了球冠模型,将液滴形态随时间的
    变化问题转化为固液接触面半径随时间的变化问题。基于球面对称性,我们对于
    接触面三相交界处的液角微元进行受力分析,在综合考虑汽-液、液-液界面液体
    自身表面内聚力、固体基板对液角微元的吸引力、液滴重力产生的压力与液滴水
    平层流之间的黏性力后对于液角微元经行受力分析。并基于学界最新提出的杨氏
    方程微观力学解释,巧妙的将三相交界处三种毛细力与杨氏方程中体现的界面
    张力建立联系以确定毛细力取值函数,并与我们推导出的压力、粘滞力函数综合
    构建出描述液角微元运动状态的动力学模型,最终通过 Runge-Kutta 迭代法编程
    求解。 
    
    在问题二中,水滴将先在空气中进行加速下落,而受粘滞阻力其撞击速度不
    确定,我们根据其释放高度对于其收下落尾速度进行讨论。此后液滴由于撞击而
    产生形变,不能再简单地视为球缺型,且不便于使用微元法分析受力情况.我们
    采用了$Navier-Stokes$公式对液滴轮廓线进行建模,并通过$Level set$法对
    $Navier-Stokes$公式进行简化处理,使用 COMSOL 软件对模型进行求解。并验
    证了第一题所建模型的正确性。
    
    在问题三中,微液滴不再与固体基质作用,而是与液体平面相作用,此文中
    我们以油层作为碰撞液体壁。与固体基质碰撞时的不同是,液体平面会在碰撞过
    程中发生形变且需要判断水滴是否进入液体基质。我们根据两种液体的密度、动
    力黏度与界面张力计算其相场变量函数的迁移率,进而确定水滴在油相当中的状
    态,最后使用 COMSOL 软件进行仿真模拟。
    
\keywords{液滴碰撞\quad  流体力学\quad   液角微元\quad  COMSOL 多相流}
\end{abstract}

\section{问题重述}
液滴在壁面的铺展过程广泛存在与自然界中,并对许多工业工程有重要意
义。 如飞机飞行过程中水滴在机翼上铺展形成液膜最终导致机翼结冰的重大安
全隐 患,或水滴在油层表面铺展形成水膜实现油层液封的工业技术。本问题中
假设液 滴在一定高度下落到壁面上,并在壁面形成铺展。在重力、表面张力、
粘性力等 作用下,液滴的铺展过程主要可分为惯性跳跃、排液/气、快速铺展和
稳定铺展等阶段在此过程中,液滴性质(如大小、粘度等)及壁面性质 (接触
角)等均会对铺展过程影响显著。     你爸爸 
 z为更好地理解液滴的铺展过程并应用到实际工业工程中,请建立数学模 
型探究液滴铺展过程中的动力学状态及关键影响因素,解决以下问题: 

1. 考虑液滴为水滴,初始时液滴位于壁面且速度为零,形状为直径 100 微
米的球形,与壁面接触角为 120 度,建立数学模型,并采用合适的数值模拟软
件计算该条件下液滴的铺展过程,获得最终液滴稳定铺展状态; 

2. 假设液滴从一定高度自由下落,从而液滴到达壁面时会存在一定的初速度
此时液滴在壁面铺展与问题 1 会存在较大差异,试探究液滴的下落速度对液滴
铺展过程的影响; 

3. 在实际应用中,很多液滴铺展的表面并非固体壁面,有可能是液体表面,
液滴在液体表面的铺展与问题 1 和问题 2 中液滴在固体壁面上的铺展存在何
种差异,考虑水滴在汽油面上的铺展过程,重新解决问题 1 和问题 2,并阐述 
该差异。
\section{问题分析}
\subsection{问题一分析}
在问题一中,由于流体铺展形变过程中的形状难以确定,且液滴尺寸较小,
重力对液滴形态影响较小,我们假设铺展过程中微液滴的形状始终为球冠状,采
用球冠模型,以三个相互关联的量 $x$ 、$theta$ 、$h$(分别为接触面半径,接触角,球
冠高度),对微液滴的形状进行表征。 
我们通过分析液滴球缺与固体壁面接触面随时间的变化来确定液滴铺展的
运动过程。在分析接触面的运动受力情况时,我们采用了微元法对接触面三相交
界处的液角微元进行受力分析。对于液角所受的三种毛细力,我们参考了
PHYSICAL REVIEW LETTERS, 并从《接触线毛细力平衡的微观成因》[1]一文中获
得启发,依据学界最新提出的杨式方程的微观力学解释,构建了界面张力和毛细
力之间的关系函数。在此基础上综合由液滴重力产生的压力、液滴水平层流之间
的黏性力,以及固体基板对液角微元的吸引力,建立了动力学模型。并编写程序
使用 Runge-Kutta 方法进行对二阶微分方程进行数值求解。
\subsection{问题二分析}
在问题二中,液滴运动主要分为两个阶段。即液滴下落但没有接触壁面的空
中阶段和液滴以一定速度撞击壁面的撞击阶段。其中,液滴撞击壁面的过程是我
们第二问主要研究的部分。 
 对于空中阶段,我们建立了简单的动力学模型,使用了密立根对斯托克斯方
程近似求解的结果,分析了液滴的收尾速度与释放高度的关系,并得出了液滴收
尾速度最大值。此后,分析第二阶段之前,我们通过第一阶段的结论得出液滴只
进行层流的结论,适当简化了第二阶段的模型。 
 对于撞击壁面阶段,我们使用了\textbf{Navier-Stokes}方程对液滴进行建模,并
使用了LSV方法,即水平集方法(Level Set)对难以直接求数值解的𝑁 − 𝑆方程进
行化简,从而求得液滴表面随时间变化的数值模拟。并且使用数值分析软件
𝐶𝑂MS𝑂𝐿内置的水平集求解器进行模型求解和验证。
\subsection{问题三分析}
在问题三中,微液滴不再与固体基质作用,而是与液体平面相作用,此题中
我们以油层作为碰撞液体壁。由于碰撞涉及水、空气、汽油三种流体介质的相互
作用,所以我们使用 COMSOL 三相流-相场模型模拟其碰撞过程。而对于水滴在油
层中的最终稳定状态,我们通过两种液体的密度、动力黏度与界面张力计算其相
场变量函数的迁移率进行定量判断。

\section{模型的假设}
1 . 假设问题一中液滴铺展过程中始终为球冠状 

2 . 假设液滴铺展过程中液体流动为水平的层流 

3 . 假设室温为 25℃,实验室内无扰动,固体壁面始终水平
\section{符号说明}
\begin{center}
\begin{tabular}{cc}
    \hline\makebox[0.3\textwidth][c]{符号} & 
    \makebox[0.4\textwidth][c]{意义} \\ 
    \hline $h$ &球缺高度\\ 
\end{tabular}
\end{center}
\section{问题分析}
\section{总结}

\bibliographystyle{plain}
\bibliography{reference}

\begin{appendices}附录的内容。
\end{appendices}
\end{document}