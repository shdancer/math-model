\documentclass{cumcmthesis}
\usepackage[framemethod=TikZ]{mdframed}
\usepackage{url}   % 网页链接
\usepackage{subcaption} % 子标题
\usepackage{multirow}
\usepackage{longtable}
\usepackage{graphicx}
\usepackage{bm}
\title{生产企业原料的订购与运输问题}\tihao{A}
\baominghao{4321}
\schoolname{你的大学}
\membera{成员A}
\memberb{成员B}
\memberc{成员C}
\supervisor{指导老师}
\yearinput{2017}
\monthinput{08}
\dayinput{22}
\begin{document}
\maketitle

\begin{abstract}
    企业生产原料的订购与运输问题是单一决策系统从众多决策方案中寻求最优解的优化问题,其目的是
    时刻保证企业利益的最大化。而为了使决策尽可能理性,往往还v需要对决策对象做出评价,寻找合
    理的评判指标与合适的评价方法将定性问题定量化。


    问题一:为了建立客观的供货商评价体系,我们查阅了诸多企业案例,确定将供应商自身供货能力
    与其供货精确性作为两大评判指标,并基于此对原料类型做\textbf{无差别化处理后}使用总供应量、单次最大供应量、供货精确的平均值与方差四项数据对其进行定量化表示。
    对于已知数据的指标,我们使用\textbf{熵权法}根据各项指标值的变异程度对其客观赋权,之后分别与\textbf{$TOPSIS$综合评价法}和\textbf{层次分析法($AHP$法)}结合,对供应商进行评分排名。
    最终求得两种方法的评价结果高度吻合,印证了模型的合理性。


    问题二:将第一小问化归为单维度优化问题后,我们采用\textbf{贪婪法}与\textbf{0-1规划方法}综合求解,得出满足企业生产需求的最少供应商数量为31家。由于未来24周这些供应商的供货量和8家转运商的运输损耗率不能确定,我们首先将过往的数据分组平均
    对其进行了模拟。此时考虑制定未来订货方案时,企业状态明确为周数和库存量,可清晰列出转移方程,故采用动态规划的思路建立模型。我们构建了\textbf{近似的动态规划模型},通过“探索-开发”行为避免决策情况的维度灾($2^{31}$),并使用分段聚类估值函数将连续值处理为离散值,得到最佳订货方案。
    而基于此方案定制最佳运输方案的优化问题可以等价为有多个背包的0-1背包问题,我们在\textbf{贪婪法}的基础上进行\textbf{0-1背包(动态)规划},得到最佳运输方案。本题模型均使用$MATLAB$编程求解,运行多次动态规划程序得出:最经济订购方案的总花费收敛在59万元附近,而最佳转运方案的损耗量为31.2881立方米


    问题三:我们将\textbf{基于分解的$MOEA\backslash D$\cite{RN13}法}应用于本题的两目标优化问题,对其的求解我们使用\textbf{切比雪夫法}建立可计算的优化公式。然后结合\textbf{遗传算法},通过遗传生成可决策域,并 使其全部遍历$MOEA\backslash D$优化后比对
    筛选出最佳权衡点,即最符合优化目标的订购方案与转运方案。得出:订购方案中原料$A$、$C$的比值为,转运方案的总损耗量为。


    问题四
    \keywords{评价类模型\quad 优化问题\quad  动态规划}
\end{abstract}

\section{问题重述}
\subsection{问题背景}
某建筑和装饰板材企业生产的周产能2.82万立方米,且生产每立方米需消耗$A$材料0.6$m^{3}$或$B$材料0.66$m^{3}$或$C$材料0.72$m^{3}$。
为满足其每年48周的生产计划,现需要根据其产能要求提前制定24周的原材料采购与转运方案:即需要确定其从“供应商”处订购的原材料数量
(即“订货量”),并且委托合适的“转运商”将“供应商”提供的原材料(即“供货量”)转运到企业仓库。($A$、$B$、$C$三种原料的运输和储存费用均相同,
而$A$、$B$的采购单价分别比$C$类原料高20 $\%$),\par
而在实际情况中,受限于原材料的特殊性,“供货量”不能完全与“订单量”吻合,但为了尽可能满足生产并减少停产风险,该企业对于供货商实际供应总是照单全收。
此外,原材料转运过程中的损耗(损耗量在供货量中的占比成为“损耗率”)会影响企业仓库实际接收原材料数量(即“接收量)。各转运商的运输能力均为
6000$m^{3}$,某供应商一周的供货量尽量由一家转运商运输。近五年402家供货商的订货量、供货量数据与转运商的运输损耗率均由附件给出。\par
\subsection{问题重述}
\begin{enumerate}
    \item [1.] 分析402家供应商的供货特征,建立数学模型,以确定50家最重要的供应商。
    \item [2.] 参考问题一,该企业至少选择多少家供应商才能保障生产?对于这些供应商,为该企业制定未来24周最经济的原料订购方案,并据此再制定使损耗量最少的转运方案。试分析两方案的实施效果。
    \item [3.] 现为减少生产成本,该企业计划尽可能多采购A原料而尽可能少采购C原料,同时还要尽量使转运损耗率减少。请为此制定订购方案及转运方案,并分析实施效果。
    \item [4.] 若该企业有能力提高产能,根据现有情况,确定其每周还能将产能提高多少,并制定未来24周的订购和转运方案。
\end{enumerate}


\section{问题分析}
\subsection{问题一分析}
求解问题一需要确定对于供应商优劣的评价机制。我们首先分析了供货商各项素质在企业订购决策中的重要程度:一方面,企业希望供应商所提供的原材料尽可能接近企业订单的需求;
另一方面,具有更好供货能力的供应商更值得企业信赖。考虑到$A$、$B$、$C$三种原料在实际生产中需求的差异,我们对三种类型的企业首先进行了无差别化处理,并综合以上考量,
构建了供应商评价机制,其中共有四个影响因素:总供应量、单次最大供应量、供货精确性(即订货量/供货量)的平均值与方差。通过熵权法可以很好的利用已知数据对四项指标客观赋权,
并将其分别与$TOPSIS$法和层次分析法结合,对供货商进行综合评价,以此排序得出前50家企业名单。对比两种方法的结果也可以检验名单的合理性。

\subsection{问题二分析}
问题二共含有三个子问题。首先要根据第一问结果选择能够满足公司生产需求的最少公司数,我们将此问题划归成了一个简单的优化问题,并且综合了两种优化方法建立满足公司生产的数学模型并对其求解:贪心
算法和迭代求解0-1规划的方法。根据求解结果确定最少商家数量供应方案中的供应商名单,并且为接下来的近似动态规划算法作预处理。此后要根据此名单制定未来24周的厂家最经济订货方案,必然要对每个周
的数据做统计学处理之后进行规划。此时题目状态明确,转移方程清晰,可以使用动态规划算法求解。但常规的动态规划方案是子集型动态规划,其状态空间异常庞大,且其状态也非离散值
而是连续值,故只能退而求其次地选择近似动态规划。近似动态规划能够相当程度减少动态规划带来的维度灾难,并在可观的时间内找到近似的最优解。最后,
当供货方案确定之后,在尽量保证供应商一周以内不切换承运商的情况下,问题化归为多个背包的0-1背包问题,其优化目标为使损耗量和函数最小。此时
使用贪心算法对背包排序,随后使用0-1背包(动态规划)算法对运货量规划,即可获得最优送货策略,其结果是精确的。

\subsection{问题三分析}
与问题二先考虑订货再考虑转运不同,问题三是一个需要同时考虑采购与订货两方面优化目标的多目标优化问题。我们采用了基于分解的多目标优化算法\cite{3},将其转化为两个单目标优化的子问题,并采用切比雪夫法建立优化公式以求得每一个
子问题的$Pareto$最优解。然后结合遗传算法,从基于原始数据一切遗传形式得来的全部$Pareto$最优解中筛选最优方案,即得到了尽量多采购A少采购C的同时又使运输损耗量最小的订购与转运方案。
\subsection{问题四分析}
问题四中,如果要提升企业的生产力,必然要在每一个瓶颈处都达到最大值。如果可以在企业-供应商-承运商之间
建立信息共享机制,我们就可以理想的认为,402家供应商的
供应量能够全部被利用,且承运商也能尽可能发挥他的能力。由于此问所需要的算法、代码在前3问已经有详尽的
叙述和编写,此问只需根据顺序设定合理的参数求解结果即可。

\section{模型的假设}
1 . 假设题中402家供货商供货供货品质均无差别且满足该企业要求。

2 . 假设不考虑货物交接流程、商家信用度、配合度、服务水平态度等细节问题

3 . 假设题目时间范围内该企业、供应商与转运商的生产能力都相对稳定,不发生重大变化。

\section{符号说明}

\begin{center}
    \begin{tabular}{cc}
        \hline\makebox[0.3\textwidth][c]{符号} &
        \makebox[0.4\textwidth][c]{意义}                                                    \\
        \hline $X_i(i=1\sim4)$                 & 供应商四种评价指标                         \\
        $X_{ij}(i=1\cdots4,j=1\cdots402)$      & 对于第i个评价指标供应商j的数据             \\
        $I_i$                                  & 信息量                                     \\
        $E_j$                                  & 信息熵                                     \\
        $w_j$                                  & 权重(由信息熵确定)                        \\
        $Z$                                    & 加权规范化矩阵                             \\
        $D^+/D^-$                              & 理想解与负理想解                           \\
        $x$                                    & 表示50家企业选择情况的向量                 \\
        $A$                                    & 50家厂商的供货情况                         \\
        $b$                                    & 根据产能推算的保障每周生产条件的原料消耗量 \\
        $I$                                    & 仓库存货量                                 \\
        $S_t(I)$                               & 第t周的仓库剩余量                          \\
        $V_t(S_t(I))$                          & 在第t周处于$S_t(I)$的总最小消耗            \\
        $C(V_t,x)$                             & 第t周在决策$x$下的该周消耗量               \\
        $T$                                    & 运输商的运输量                             \\
        \hline
    \end{tabular}
\end{center}
\section{模型的建立与求解}
\subsection{问题一模型的建立与求解}

\subsubsection{评价指标的确定}
量化分析供货商供货特征并做出评价首先应确定可以量化的相应指标,在题干语境下
我们就两大方面对供货商进行评价:供货量的精确度与供货商本身的供货能力,下面分别作出分析
\subsubsection*{供货精确度}
由于企业采取照单全收的方案,在给供货商下达一个订单时,企业自然希望供货商届时生产的原材料量
与企业所给的订单量尽可能接近:如果供货量远少于订货量,则企业由于缺少必要原料无法正常按计划生产
,从而造成收益降低;如果供货量远大于订单量,则企业收购过量的原料会产生不必要的持货成本和空间浪费。考虑了实际的供货数据,我们用订货量与供货量的比值表示供应商的供货精确度。该精度度越接近1
则企业越满意,通过计算其\textbf{方差和均值},就可以对供货精确度进行量化分析。

\subsubsection*{供货能力}
统计分析了402家供货商过去五年内的供货记录,我们发现不同供货商间的供货能力存在较大差异,而那些具有较强
原料生产能力的供货商与企业的订单合作更为频繁,这也说明了供货能力更强的大供货商地位更为重要。由于不同供货商的
供货特征不同(有些以一定数量稳定供货而有些以一定周期大批量供货),我们从\textbf{单次最大供货量与供货总量}两个维度来
综合的定量衡量企业的供货能力。\par
%图形化表示供货商特征
\begin{figure}[htbp]
    \centering
    \includegraphics[scale=0.6]{offer.png}
    \caption{部分企业前五年的供货情况}     \label{fig:1}
\end{figure}

为了便于表述,后文中使用符号$X_1$、$X_2$、$X_3$和$X_4$来分别表示精确度的平均值、方差、供货总量与单次最大供货量这四个评价
指标。

\subsubsection{使用熵权法为指标客观赋权}
熵值法是一种根据各项指标指标值的变异程度来确定指标权数的客观赋权法。无论是选择$TOPSIS$法还是层次分析法,首先都需要得到评价指标的具体权重。$TOPSIS$法常常需要邀请专家对于评价指标打分,而层次
分析法则是通过两两比较各指标赋予其重要性标度来推算权重。这两种方法的弊病就是\textbf{赋权过程具有较大主观性},而基于客观数据信息量的
熵权法可以\textbf{基于指标的变异性大小客观赋权},很好的弥补了这一劣势。
%角标
\subsubsection*{熵权法的实现原理}
熵权法中的”熵“即信息熵,它指信息量的数学期望。这里的信息量是信息论中基于事物出现概率的概念
\begin{equation}
    I_i = log_2(\frac{1}{P_i})=-log_2P_i
    \label{信息量}
\end{equation}
由公式可见其与某事件出现的概率成反比。从意义上理解,信息量即将一个事物确定下来需要查询的信息的量。当某事物信息熵较小时,出现概率较大,
其就具有了较高的确定性(这意味着关于该事物的未知信息少,已知信息多),该事物所能提供的信息量就越大,那么它在综合评价体系中就具有了较大权重。
由其具有确定的计算公式,我们可以根据信息熵客观的为各个指标定量衡量权重。
\subsubsection*{熵权法赋权模型的建立与求解}
使用熵权法为指标客观赋权,我们首先将附录数据正向化,并进行了归一化处理
\begin{equation}
    Y_{ij}=\frac{X_{ij}-min(X_i)}{max(X_i)-min(X_i)}\large\nonumber
    \label{归一化}
\end{equation}
其中,对于$X_{ij}$,$i$取值介于$1\sim4$,表示四项评价指标。$j$取值介于$1\sim402$,表示各项指标对于每家供应商
的具体数据。由附表信息,$X_{ij}$的取值可以很容易得出。\par
利用正向标准化后的数据得到所有情况中各指标的比重(变异性),即前文所说的某事件出现的”概率“
\begin{equation}
    p_{ij}=\frac{Y_{ij}}{\sum_{i=1}^{n}{Y_{ij}}},i=1,\cdots,n,j=1,\cdots,m\nonumber
    \label{指标比重}
\end{equation}
将其代入信息熵的计算公式
\begin{equation}
    E_j = -\ln(n)^{-1}\sum_{i=1}^{n}{p_{ij}}\ln{p_{ij}}\nonumber
\end{equation}
这样就确定了对应于信息熵的已知信息量,各指标信息量的比重就客观反映了其权重
\begin{equation}
    w_j=\frac{1-E_j}{m-\sum E_j}(j=1,2,\cdots,m)
\end{equation}
\subsubsection*{熵权法程序设计}
值得注意的一点是,由于在实现熵权法的过程中,需要经过归一化、计算概率等相关处理,所以在实际程序编写的
过程中不可避免的会遇到概率值为0的情况。而在计算信息量时,由(1)式可知,其在0处没有定义。
(这是显而易见的,因为概率为零的信息量无穷大。所以在编程的过程中需要对归一化之后的数值经行修正,给其加上
一个很小的常数$\epsilon$\cite{黄鹏2015福建省土地生态安全AHP法和熵值法动态评价比较},从而使其有意义。\par
%%%%%%%%%%%%%%%%%%%%%%%%%%%%%%%%%%%%%%引用
熵权法为四个指标客观赋权的程序框图如下
\begin{figure}[H]
    \centering
    \includegraphics[scale=0.6]{shangquan.png}
    \caption{熵权法程序框图}     \label{fig:2}
\end{figure}
\subsubsection{结合$TOPSIS$法的综合评价}
\subsubsection*{$TOPSIS$法概述}
TOPSIS法(又称为”优劣解距离法“)是根据有限个评价对象与理想化目标的接近程度进行排序进而评价现有目标相对优劣
性的方法。其适用范围广泛,多应用于有量化指标的评价类问题上。其原理是检测评价对象对于”理想解“和”负理想解“
之间的距离,并以此为依据对各方案综合评分,排序确定评价对象的相对优劣。但其\textbf{缺点是权重值的得出通常具有主观性},导致
评价结果说服力不足。而此题\textbf{我们优化了TOPSIS评价模型,将其与熵权法结合,通过客观赋权增强其结论的说服力。}
\subsubsection*{TOPSIS法评价模型的建立}
对于有4个属性,402个评价对象的体系构建原始数据矩阵
\begin{equation}
    X=\begin{bmatrix}
        X_{11}   & X_{12}   & \dots  & X_{14}   \\
        X_{21}   & X_{22}   & \dots  & X_{24}   \\
        \vdots   & \vdots   & \ddots & \vdots   \\
        X_{4021} & X_{4022} & \dots  & X_{4024} \\
    \end{bmatrix} \quad
    \nonumber
    \label{数据矩阵}
\end{equation}
对该矩阵进行向量标准化处理,得到规范化矩阵$z$
\begin{equation}
    z_{ij}=\frac{X_{ij}}{\sqrt{\sum\limits^n_{i=1}X^2_{ij}}}\nonumber
    \label{规范化矩阵}
\end{equation}
结合$(1)$式熵权法确定的权重,构造加权规范化矩阵
\begin{equation}
    Z=(Z_{ij})_{(n\times m)}=z_{ij}\cdot w_j\nonumber
    \label{加权规范化矩阵}
\end{equation}
根据加权规范化矩阵,确定其”理想解“与”负理想解“
\begin{equation}
    Z^+ = (max\{Z_{11},Z{_21},\dots,Z_{n1}\},max\{Z_{12},Z_{22},\dots,Z_{n2}\},max\{Z_{1m},Z_{2m},\dots,Z_{nm}\}) =(Z^+_1,Z^+_2,\dots,Z^+_m)\nonumber
\end{equation}
\begin{equation}
    Z^+ = (min\{Z_{11},Z{_21},\dots,Z_{n1}\},min\{Z_{12},Z_{22},\dots,Z_{n2}\},min\{Z_{1m},Z_{2m},\dots,Z_{nm}\}) =(Z^-_1,Z^-_2,\dots,Z^-_m)\nonumber
\end{equation}
并以此计算最优、最劣距离
\begin{equation}
    D^+=\sqrt{\sum_{j=1}^{n}{(Z_{ij}-Z^+_j)^2}}\nonumber
\end{equation}
\begin{equation}
    D^-=\sqrt{\sum_{j=1}^{n}{(Z_{ij}-Z^-_j)^2}}\nonumber
\end{equation}
得到理想贴进度
\begin{equation}
    C_i=\frac{D^-_j}{D^+_j+D^-_j}\nonumber
\end{equation}
进行排序,选出前50名即为最重要的50家供货商。
\subsubsection*{TOPSIS法评价模型程序求解流程}
由于TOPSIS法中矩阵的规范化等价于正向化与归一化两个条件,因此我们编写程序时直接使用了熵权法已经处理过的正向归一化
矩阵以简化运算,程序流程如下:
\begin{description}
    \item[$\blacktriangleright$] \textbf{Step1} 读取规范化矩阵与权重表
    \item[$\blacktriangleright$] \textbf{Step2} 将规范化矩阵与熵权法得到的权重表运算得到加权规范化矩阵$Z$
    \item[$\blacktriangleright$] \textbf{Step3} 遍历$Z$得到“理想解“与”负理想解”
    \item[$\blacktriangleright$] \textbf{Step4} 计算最优、最劣距离并将结果排序
    \item[$\blacktriangleright$] \textbf{Step5} 输出前50名供货商
\end{description}
\begin{description}
    \item[$\bigstar$] 结果展示:
\end{description}

\begin{longtable}{l|llll|l}
    \toprule
    编号 & 精确度平均值 & 精确度的方差 & 供货总量 & 单次最大供货量 & TOPSIS综合得分 \\
    \midrule
    140  & 1.199825272  & 0.160545743  & 302047   & 21293          & 0.675198902    \\
    348  & 1.070133152  & 0.09454509   & 92421    & 36972          & 0.622301599    \\
    151  & 0.983328206  & 0.020063101  & 194498   & 21267          & 0.564549137    \\
    201  & 0.536341199  & 0.254117575  & 81989    & 30977          & 0.563314813    \\
    229  & 0.994426055  & 0.004697454  & 354887   & 3147           & 0.480728969    \\
    361  & 0.992053674  & 0.005377459  & 328080   & 2816           & 0.458163546    \\
    374  & 1.312158559  & 0.112746088  & 49224    & 23695          & 0.446672752    \\
    108  & 0.984082869  & 0.01704667   & 240950   & 7885           & 0.423816598    \\
    139  & 1.243990587  & 0.097592391  & 151862   & 10207          & 0.343906103    \\
    330  & 0.985590135  & 0.019036115  & 136652   & 9768           & 0.319015574    \\
    126  & 0.989970416  & 0.186240219  & 47540    & 15114          & 0.312548133    \\
    308  & 0.974520012  & 0.028175881  & 136998   & 8181           & 0.297871994    \\
    282  & 1.013308551  & 0.004372127  & 169340   & 1724           & 0.279658746    \\
    340  & 0.999293707  & 0.001889266  & 171426   & 1181           & 0.279233345    \\
    275  & 1.003392888  & 0.000306742  & 158553   & 966            & 0.260688259    \\
    329  & 1.0019197    & 0.000537446  & 156518   & 971            & 0.257928479    \\
    307  & 1.244248204  & 0.16870653   & 78196    & 9385           & 0.241823064    \\
    131  & 0.997395498  & 0.009990159  & 137512   & 1014           & 0.231473631    \\
    356  & 1.002613641  & 0.005963545  & 130307   & 1788           & 0.225389591    \\
    268  & 1.00351123   & 0.000539785  & 129786   & 736            & 0.219065135    \\
    306  & 1.004597407  & 0.000568041  & 126096   & 922            & 0.214575308    \\
    395  & 0.918094209  & 0.120756485  & 75843    & 7661           & 0.2123179      \\
    194  & 1.00196034   & 0.001948659  & 101365   & 595            & 0.176387837    \\
    143  & 1.003667731  & 0.038807572  & 82787    & 2521           & 0.159062768    \\
    352  & 1.001534168  & 0.004841805  & 89031    & 699            & 0.157983582    \\
    37   & 1.14974634   & 0.158054432  & 50686    & 5398           & 0.147906884    \\
    247  & 1.013939114  & 0.000884787  & 56698    & 342            & 0.106691293    \\
    284  & 1.0381543    & 0.002036033  & 46597    & 2005           & 0.101056396    \\
    365  & 1.021696663  & 0.002089781  & 41631    & 381            & 0.083795136    \\
    31   & 1.019774282  & 0.003261046  & 41207    & 251            & 0.082807204    \\
    338  & 1.264371563  & 0.111733709  & 30109    & 2081           & 0.076270086    \\
    40   & 1.018751142  & 0.01216166   & 31905    & 432            & 0.069710381    \\
    364  & 1.038518536  & 0.009651035  & 28763    & 618            & 0.065988507    \\
    55   & 1.039824474  & 0.040958131  & 24041    & 1216           & 0.063155977    \\
    367  & 1.030694449  & 0.022426908  & 26335    & 597            & 0.062481832    \\
    346  & 1.016227876  & 0.017439483  & 23240    & 162            & 0.057352717    \\
    294  & 1.042035369  & 0.00644309   & 18842    & 113            & 0.051993084    \\
    86   & 1.195310114  & 0.179352469  & 17949    & 1265           & 0.051884414    \\
    80   & 1.086134053  & 0.040086494  & 19237    & 215            & 0.051179135    \\
    244  & 1.071080527  & 0.046899012  & 16406    & 180            & 0.047954646    \\
    218  & 1.098141397  & 0.024692137  & 15483    & 180            & 0.047237803    \\
    74   & 1.006205778  & 0.232847768  & 13051    & 1043           & 0.046185824    \\
    210  & 0.832213153  & 0.241177511  & 15694    & 1034           & 0.046177575    \\
    3    & 1.059890496  & 0.113518145  & 13138    & 387            & 0.043638188    \\
    114  & 1.168151472  & 0.116036871  & 10931    & 721            & 0.041843506    \\
    273  & 0.993120594  & 0.135664811  & 9484     & 537            & 0.041576485    \\
    189  & 1.016986759  & 0.086221784  & 8892     & 116            & 0.04083377     \\
    78   & 0.990774338  & 0.122525234  & 8553     & 412            & 0.040602391    \\
    5    & 1.04055103   & 0.09553314   & 6912     & 128            & 0.038895059    \\
    291  & 1.149152262  & 0.14595773   & 7984     & 686            & 0.038785968    \\
    \bottomrule
\end{longtable}

\subsubsection{结合层次分析法的综合评价}

\subsubsection*{层次分析法概述}
层次分析法是一种分层次的权重决策分析方法,善于将定性问题定量化。其本质是用分层分析的思想将多目标决策问题拆解为多个影响因素之间的
定量分析。本题中评价供货商重要与否就同样可以使用层次分析法的思考方式。在此基础上,我们结合了熵权法得出的指标权重,且由于已知各项指标的具体数据
已知,我们省去了层次分析法中两两比较主观赋权的过程,而是\textbf{利用客观数据反映判断矩阵的标度值,极大的避免了层次分析法的盲目性与不确定性。}
此方法也可以为TOPSIS法的结果作出印证。
\subsubsection*{层次分析法模型的建立与求解}
\begin{description}
    \item[$\blacktriangleright$] \textbf{建立层次结构}\par
        按层次分析法将决策系统自上而下分为目标层、准则层与方案层三个层次,层次结构如图所示:
        \begin{figure}[htbp]
            \centering
            \includegraphics[scale=0.6]{AHP.png}
            \caption{层次分析法示意图}     \label{fig:3}
        \end{figure}
        \\
        \\
    \item[$\blacktriangleright$] \textbf{构造判别矩阵}\par
        对于各项指标的权重关系,我们采用了熵权法的结果,避免了赋权的主观性。而确定各指标对于供应商的权重时,我们选择采用算术平均、几何平均与特征值法三种方法综合,对
        前面计算中得到的各项指标的客观数据求权重,同样避免了主观赋权对于真实结果的影响。\par
        综合每一个供货商关于所有指标的计算结果,得到最终的权重矩阵\par
        \begin{table}[htbp]
            \begin{tabular}{|l|l|l|l|l|}
                \hline
                                         & 指标权重    & S001        & S002        & ...      \\
                \hline
                准确度均值               & 0.01915795  & 0.337362637 & 0.919749373 & $\cdots$ \\
                \hline
                准确度方差               & 0.021462962 & 0.381477411 & 0.433076045 & $\cdots$ \\
                \hline
                总供货量                 & 0.438240822 & 49          & 273         & $\cdots$ \\
                \hline
                准确度均值单日最大供货量 & 0.521138266 & 6           & 67          & $\cdots$ \\
                \hline
            \end{tabular}
        \end{table}
        将供应商各项权重列与指标权重列一一对应相乘并求和,得到基于熵权法与层次分析法评价体系的供货商最终得分。

    \item[$\bigstar$] \textbf{层分析法求解结果展示:}
        \begin{longtable}{l|llll|l}
            \toprule
            编号 & 精确度平均值 & 精确度的方差 & 供货总量 & 单次最大供货量 & TOPSIS综合得分 \\
            \midrule
            140  & 1.199825272  & 0.160545743  & 302047   & 21293          & 0.704336915    \\
            348  & 1.070133152  & 0.09454509   & 92421    & 36972          & 0.671265896    \\
            151  & 0.983328206  & 0.020063101  & 194498   & 21267          & 0.579546338    \\
            201  & 0.536341199  & 0.254117575  & 81989    & 30977          & 0.560696564    \\
            229  & 0.994426055  & 0.004697454  & 354887   & 3147           & 0.522947876    \\
            361  & 0.992053674  & 0.005377459  & 328080   & 2816           & 0.485106422    \\
            108  & 0.984082869  & 0.01704667   & 240950   & 7885           & 0.448400194    \\
            374  & 1.312158559  & 0.112746088  & 49224    & 23695          & 0.425390897    \\
            139  & 1.243990587  & 0.097592391  & 151862   & 10207          & 0.36387778     \\
            330  & 0.985590135  & 0.019036115  & 136652   & 9768           & 0.346099315    \\
            308  & 0.974520012  & 0.028175881  & 136998   & 8181           & 0.32362686     \\
            126  & 0.989970416  & 0.186240219  & 47540    & 15114          & 0.305781555    \\
            282  & 1.013308551  & 0.004372127  & 169340   & 1724           & 0.273603295    \\
            340  & 0.999293707  & 0.001889266  & 171426   & 1181           & 0.268857797    \\
            307  & 1.244248204  & 0.16870653   & 78196    & 9385           & 0.258882151    \\
            275  & 1.003392888  & 0.000306742  & 158553   & 966            & 0.249930644    \\
            329  & 1.0019197    & 0.000537446  & 156518   & 971            & 0.247509025    \\
            395  & 0.918094209  & 0.120756485  & 75843    & 7661           & 0.236502238    \\
            356  & 1.002613641  & 0.005963545  & 130307   & 1788           & 0.226456584    \\
            131  & 0.997395498  & 0.009990159  & 137512   & 1014           & 0.224307037    \\
            268  & 1.00351123   & 0.000539785  & 129786   & 736            & 0.211151881    \\
            306  & 1.004597407  & 0.000568041  & 126096   & 922            & 0.209194359    \\
            143  & 1.003667731  & 0.038807572  & 82787    & 2521           & 0.176960759    \\
            194  & 1.00196034   & 0.001948659  & 101365   & 595            & 0.174047598    \\
            37   & 1.14974634   & 0.158054432  & 50686    & 5398           & 0.170928484    \\
            352  & 1.001534168  & 0.004841805  & 89031    & 699            & 0.160191009    \\
            284  & 1.0381543    & 0.002036033  & 46597    & 2005           & 0.125571674    \\
            247  & 1.013939114  & 0.000884787  & 56698    & 342            & 0.115119818    \\
            338  & 1.264371563  & 0.111733709  & 30109    & 2081           & 0.098089909    \\
            365  & 1.021696663  & 0.002089781  & 41631    & 381            & 0.096868662    \\
            31   & 1.019774282  & 0.003261046  & 41207    & 251            & 0.094510349    \\
            40   & 1.018751142  & 0.01216166   & 31905    & 432            & 0.085290168    \\
            55   & 1.039824474  & 0.040958131  & 24041    & 1216           & 0.085232121    \\
            364  & 1.038518536  & 0.009651035  & 28763    & 618            & 0.083729036    \\
            367  & 1.030694449  & 0.022426908  & 26335    & 597            & 0.08015199     \\
            346  & 1.016227876  & 0.017439483  & 23240    & 162            & 0.070652613    \\
            86   & 1.195310114  & 0.179352469  & 17949    & 1265           & 0.070618444    \\
            294  & 1.042035369  & 0.00644309   & 18842    & 113            & 0.064398908    \\
            80   & 1.086134053  & 0.040086494  & 19237    & 215            & 0.064309174    \\
            74   & 1.006205778  & 0.232847768  & 13051    & 1043           & 0.063329308    \\
            210  & 0.832213153  & 0.241177511  & 15694    & 1034           & 0.063007103    \\
            244  & 1.071080527  & 0.046899012  & 16406    & 180            & 0.060382775    \\
            218  & 1.098141397  & 0.024692137  & 15483    & 180            & 0.059469472    \\
            3    & 1.059890496  & 0.113518145  & 13138    & 387            & 0.057209759    \\
            114  & 1.168151472  & 0.116036871  & 10931    & 721            & 0.056978949    \\
            273  & 0.993120594  & 0.135664811  & 9484     & 537            & 0.055096933    \\
            78   & 0.990774338  & 0.122525234  & 8553     & 412            & 0.052587767    \\
            291  & 1.149152262  & 0.14595773   & 7984     & 686            & 0.052197741    \\
            189  & 1.016986759  & 0.086221784  & 8892     & 116            & 0.049921203    \\
            5    & 1.04055103   & 0.09553314   & 6912     & 128            & 0.046864135    \\
            \bottomrule
        \end{longtable}
\end{description}

\subsection{问题二模型的建立与求解}
\subsubsection{确定最少数量供货商}
我们使用了贪心算法和0-1规划两种思路分别求解满足生产需求所需的最少供应商数量。
此外,出于将复杂问题拆解为小问题的考虑并且为之后的动态规划计算做铺垫,
我们还对50个重要供货商进一步排序使得他们更加契合本问的要求。
\subsubsection*{贪心算法}
\begin{description}
    \item[$\blacktriangleright$] \textbf{贪心算法概述}\par
        对于一个具有最优子结构的问题,可以对其使用贪心算法求解。结合熵权法分析得出,供应商的供货能力是影响其评价体系的主要因素。对于此问题,如果以过去五年的平均供货量作衡量供应商
        对厂商生产需求的重要程度的指标,任意供应商$A,B$的平均供货量在排序交换之后是没有变化的,这时我们也称
        以平均供货量作为评价标准下,选择供应商具有最优子结构。此时只需要\textbf{贪心地根据平均供货量从大到小选择供应商直至满足企业生产需求即可}。但
        要注意的是,在求解平均供货量数据中由于厂家未下订单的数据不具参考价值,应当剔除!
    \item[$\blacktriangleright$] \textbf{贪心算法程序设计}\par
        如图\ref{fig:4}所示,贪心算法主要由判断、计算、得出结果三部分组成。其中的计算与得出结果都比较容易。
        而判断其是否具有最优子结构往往是不容易想到的,此题中供货商之间的平均供货量彼此无关且顺序不影响指标,所以其为最优子结构。
        \begin{figure}[H]
            \centering
            \includegraphics[scale = 0.6]{TX.png}
            \centering
            \caption{贪心算法流程} \label{fig:4}
        \end{figure}
\end{description}
\subsubsection*{0-1规划}
\begin{description}
    \item[$\blacktriangleright$] \textbf{0-1规划概述}\par
        在规划问题中,若待规划向量的取值只能取0和1,则其称为0-1规划问题。
        考虑线性规划问题都有迭代解法,所以使用$MATLAB$的$linprog$函数可以方便的
        解决此类线性规划问题。
    \item[$\blacktriangleright$] \textbf{0-1规划模型建立}\par
        在此题中,一个厂商要不要选择该供应商作为供货方案是一个0-1变量,由于
        使用0-1规划,我们可以细粒度的通过每一周厂商的历史供货情况做一个规划。
        设50家企业的选择情况是向量$x$,厂商的供货情况用矩阵$A$进行表示($A_{i,j}$表示$i$厂商
        第$j$周的供货情况)。
        \begin{equation}
            x_i = \left\{
            \begin{aligned}
                0,\quad & x \in notChosen \\
                1 \quad & x \in Chosen
            \end{aligned}
            \right.,\quad i=0,1,\cdots,50 \nonumber
        \end{equation}
        又因为为了尽量保证企业生产,要尽量留有够生产两周的库存量,设两周的制作量为$b$,有限定条件:
        \begin{equation}
            \begin{aligned}
                \sum_{i=1}^{50} A_{i,j} \cdot x_i & + I_{j-1}  > b                                             \\
                where. I_{j}  = I_{j-1}           & + \sum_{i=1}^{50} A_{i,j} \cdot x_i + I_{j-1} -b \nonumber
            \end{aligned}
        \end{equation}
        由于$I$是一个递推公式,经过化简可以得到矩阵形式的限定条件:
        \begin{equation}
            A^{*} \bm{x} > \bm{b^{*}} \nonumber
        \end{equation}
        {\small
        \begin{equation}
            \begin{aligned}
                A^* =
                \begin{pmatrix}
                    A_{1,1}                                & A_{1,2}                                & \cdots & A_{1,50}                                  \\
                    A_{1,1} + A_{2,1}                      & A_{1,2} + A_{2,2}                      & \cdots & A_{1,50} + A_{2,50}                       \\
                    \vdots                                 & \vdots                                 & \ddots & \vdots                                    \\
                    A_{1,1} + A_{2,1} + \cdots + A_{240,1} & A_{1,2} + A_{2,2} + \cdots + A_{240,2} & \cdots & A_{1,50} + A_{2,50} + \cdots + A_{240,50}
                \end{pmatrix} \nonumber
            \end{aligned}
        \end{equation}
        }
        \begin{equation}
            b^*_i = b_i \cdot i \nonumber
        \end{equation}
        所以规划问题总体表述为:
        \begin{equation}
            \begin{aligned}
                \min_x \bm{e^t}\cdot \bm{x} \\
                s.t. \left\{
                \begin{aligned}
                    A^*\bm{x} > b^* \\
                    x_i \in intcon
                \end{aligned}
                \right.
            \end{aligned} \label{规划问题表述}
        \end{equation}

    \item[$\blacktriangleright$] \textbf{0-1规划程序设计}\par
        由式\ref{规划问题表述},我们直接调用$MATLAB$中的$linprog$函数进行
        0-1线性规划。
    \item[$\bigstar$] \textbf{最少数量供货商结果求解}\par
        我们首先尝试了使两种算法满足企业现有产能2.82万立方米/周,发现若供应商供应能力不变的话总供应额上限只能达到2.1万立方米/周左右,小于企业现有生产能力,故我们认为企业应仍保持原来生产水平。
        我们基于企业过往订单的供货量评估了其真实产能,并在此基础上重新进行优化,得到满足企业生产所需最少供应商数量为31家。


\end{description}
\subsubsection{近似动态规划确定未来24周的最佳订货方案}
在挑选出恰当的供应商后,还要对供应商以周为单位确定订货方案
,经过分析,我们参考了文献\cite{RN10},使用了近似动态规划的方法求解问题。
\subsubsection*{动态规划概述}
动态规划是求解规划问题最强有力的工具之一。他包含分治和状态机
的思想,对问题进行拆分,解决并合并,得到优化问题的最优解。
使用动态规划有几点要求:
\begin{itemize}
    \item \textbf{无后效性} \quad 优化子问题时不影响父问题。
    \item \textbf{具有最优子结构} \quad 最优策略的子策略总是最优的。
    \item \textbf{状态空间重叠} \quad 动态规划需要储存大量状态,若状态空间巨大则无法达到目的。
\end{itemize}
我们可以看出,此题的结构适合使用动态规划算法。\par
\begin{description}
    \item[$\blacktriangleright$] \textbf{状态和状态函数} \\
        由于我们需要对每周构造合理的方案,所以其状态必然包含周数;
        其次,库存量是重要的维度之一,此时也要设计进入状态:
        $$
            S_t(I)
        $$
        表示$t$周之前都满足生产且$t$周是仓库剩余量为$I$的状态。我们在这里直接设要优化
        的问题作为状态函数$V$,即:
        $$
            V_t(S_t(I)) = V_t(I)
        $$
        $V$表示该状态最小需要花费的钱,至此状态和状态函数设计完毕。(由于花的钱为相对值,所以我们假设他的单位是1)
    \item[$\blacktriangleright$] \textbf{转移方程} \\
        动态规划的核心之一是状态转移方程(Bellman)方程,向后逐步求解Bellman方程是常规动态规划的基本思路
        ,我们考虑从一个周转移到下一个周的一个\textbf{决策}$\bm{x}$,其转移方程为:
        \begin{equation}
            V_t(I) = V_{t-1}(I - \sum \bm{x} + b) + C(V_t,\bm{x}) \quad t = 1,2,\cdots , n-1
            \label{状态转移方程}
        \end{equation}
    \item[$\blacktriangleright$] \textbf{边界条件} \\
        动态规划求解的时候,需要边界条件约束,才能将
        题目的状态空间限定到有限且符合题意的范围中。此题中边界条件为:
        $$
            \left\{
            \begin{aligned}
                V_1(0) = 0 \\
                I > 0
            \end{aligned}
            \right.
        $$
\end{description}
\subsubsection*{动态规划的不足和近似动态规划的引入}
动态规划在处理这种阶段性问题时候有得天独厚的优势。但也有不足之处,例如
动态规划在处理\textbf{决策$\bm{x}$}的时候,由于其为子集型动态规划,若要
枚举尽所有的可能性,则每步\textbf{有$2^{31}$种可能性},此时这会出现成为维度灾的时间复杂度
急速膨胀的情况。其次,状态的分量$I$是一个\textbf{连续的变量而不是离散的变量},故使用
动态规划则会出现状态空间过于庞大的情况。\par
因此我们使用了名为近似动态规划\cite{RN10}的动态规划改进算法,用于处理维度灾和非连续分量。
近似动态规划主要用,\textbf{分段聚类估值函数}处理连续变量值,用\textbf{探索-开发行为}进行决策的缩减
,他完美的解决了动态规划在具体实现上的难点。其次,作为随机算法,近似动态规划
收敛快,能够较好的收敛到精确最优解。

\subsubsection*{近似动态规划模型建立}
建立近似动态规划模型时,要进一步细化动态规划的细节,并且对完善的动态规划进行
近似动态规划的特殊处理。
\begin{description}
    \item[$\blacktriangleright$] \textbf{预测供货量} \\
        制定采购方案即在选择出的31家所需最少供应商其中制定方案。选择供货商最的判断依据就要对每个周的供货量进行量化。
        过往数据表明,在同一时间,不同供货商供货量各不相同。因此,未来24周的供货量率不能明确确定,需要做出预测。我们对过往的240组数据均分为24组,用每组供货量的平均水平模拟各供货商未来24周每周平均的供货量。
    \item[$\blacktriangleright$] \textbf{价格函数} \\
        由于购买材料时,基本不受外界因素影响,只与材料种类有关,所以$C(V_t,\bm{x})$
        可以简化为$C(\bm{x})$:
        $$
            C(\bm{x}) = \bm{c}^{\text{T}}\bm{x}
        $$
        其中$\bm{c}^{\text{T}}$是各公式售卖对应材料的价值和体积的综合常量值。
    \item[$\blacktriangleright$] \textbf{最近聚合估值} \\
        为了解决连续变量问题,我们需要使用估值算法将某些计算对式\ref{状态转移方程}进行简化:
        $$
            V_t(I) = \overline{ V }_{t-1} (I - \sum \bm{x} + b) + C(V_t,\bm{x})
        $$
        其中$\overline{V}_{t-1}$为$V_{t-1}$的估计值,也称作观测值。
        通常我们的估计函数有聚合法,表查询法,基函数法。而聚合法常用的又有分段聚合法,多层聚合法,
        最近聚合法。由于我们在前面已经对402个供货商进行了两轮筛选,剩下的供货商优秀度高,所以我们
        采用编写更简单的最近聚合法。他的表述为:
        $$
            \overline{V}_t(I) = H(V_t(I)) = V_{t-1}(\left. X \right| \min_{dist(I,X)})
        $$
        即取离$I$最近的已有的值作为他的观测值。
    \item[$\blacktriangleright$] \textbf{探索-开发行为} \\
        除连续变量问题之外,由于厂商只有照单收买和不下单两种选择,
        故此动态规划是子集型动态规划,他每一步决策数量都十分庞大。
        近似动态规划中,主要以探索-开发分别进行,以参数控制比例。但此问中显然不能进行开发行为,
        因为任意一次开发都有高达$2^{31}$种情况。但是若是纯粹进行探索
        的话,很容易使算法进入次优解而失去动态规划的意义。\par
        所以我们采用了探索-开发行为,即先对决策进行随机,之后再根据随机策略进行开发。
        这样既能兼顾开发能搜索到更多状态,也避免收敛到次优解。
\end{description}

\subsubsection*{近似动态规划程序流程}
如图\ref{fig:6},近似动态规划程序流程稍显复杂,需要一定编码技巧。
在实际实现中,使用递归的方法而不是迭代的方法进行编写。
\begin{figure}[H]
    \centering
    \includegraphics[scale = 0.5]{GH.png}
    \centering
    \caption{近似动态规划算法图} \label{fig:6}
\end{figure}


\subsubsection{制定损耗最少的转运方案}
此小问仍然可以等价为一个最优化问题。由于损耗律不确定,我们先模拟出了未来24周的损耗率情况,并\textbf{将贪婪法与0-1背包(动态规划)模型结合建立制定最佳转运方案的模型},最后
用$MATLAB$求解出最佳转运方案。
\subsubsection*{制定最佳转运方案模型的建立}
\begin{description}
    \item[$\blacktriangleright$]\textbf{预测损耗率}  \\
    制定转运方案即为我们已经得出的未来24周最优订购方案选择转运商。由于各转运商每周运输能力均相同。因此,选择转运商最主要的判断依据就是在原材料转运过程中不同转运商的原材料损耗率。
    过往数据表明,在同一时间,不同转运商损耗率各不相同,且不同时间相同转运商损耗率也各不相同。因此,未来24周的损耗率不能明确确定,需要做出预测。我们对过往的240组数据均分为24组,用每组损耗率的平均水平模拟各转运商未来24周每周的损耗率。
    \item[$\blacktriangleright$]\textbf{0-1背包(动态规划)模型与贪婪法结合}\\
    现在此问题的已知条件变为了
    \begin{itemize}
        \item 有确定的订货方案,其中各订货商数量、订货数量均为已知
        \item 有8家可选择的转运商,其中各转运商转运能力已知、运输每周各运输商运输损耗率虽不同但已知
        \item 一个供应商每周尽量选择一家转运商,而一家转运商每周可以运输多家供应商的货物。
    \end{itemize}
    而我们要做的是确定每周使损耗最少的运输方案。首先,可以利用贪婪算法对每周八家供货商的损耗率排序。现考虑单独一周的运输量分配问题,对于这周损耗率最小的供应商,考虑两种一般的转运商选择方案。假设其分配到该转运商的运输量
    \begin{equation}
        T_{1j}<T_{2j}\nonumber
    \end{equation}
    其中$T$的第一个角标表示这两种一般方案,第二个角标表示转运商代号。那么由于两种方案的总运输量相同,一定对于另外一个转运商存在这种情况:
    \begin{equation}
        T_{1k}>T_{2k}\nonumber
    \end{equation}
    若这两种方案除此之外对于转运商的选择完全相同,假设用$a_{x}$表示不同转运商的损耗率,那么可以得出两种方案总损耗量的关系
    \begin{equation}
        \Delta Waste= Waste_1-Waste_2=a_i T_{1j}+ a_j T_{1k}-a_i T_{2j} + a_j T_{2k}=(a_i-a_j)\vartriangle T\nonumber
    \end{equation}
    其中最右边两项均为负数,可知 $\Delta Waste >0$,因此方案二损耗量更小,或者说将方案一的状态转移到方案二损耗量更小。这也意味着如果想要使运输损耗量尽可能的小,应该尽可能利用损耗率小的商家进行运输。此时该问题对于单独一个转运商就转化为了0-1背包问题,
    接下来只需在贪婪法得出的转运商损耗率排名名单上从小到大使用0-1动态规划即可。
\end{description}
\subsubsection*{制定最佳转运方案模型的程序求解}
\begin{description}
    \item[$\blacktriangleright$] \textbf{Step1} 读取转运商数据与订购方案数据。
    \item[$\blacktriangleright$] \textbf{Step2} 对转运商数据分组处理,得到未来24周损耗率模拟值
    \item[$\blacktriangleright$] \textbf{Step3} 使用贪婪算法得出损耗率从小到大的转运商排序名单
    \item[$\blacktriangleright$] \textbf{Step4} 使用0-1背包(动态规划)算法
        \begin{itemize}
            \item \textbf{Step4.1}遍历所有天数
            \item \textbf{Step4.2}遍历损耗率序列
            \item \textbf{Step4.3}使用状态转移方程
        \end{itemize}
    \item[$\blacktriangleright$] \textbf{Step5} 输出转运方案
\end{description}



\subsection{问题三模型的建立与求解}
\subsubsection{问题三模型的建立}
本问仍然涉及对订购方案和转运方案进行优化,但由于在满足订购方案优化目标(即多采购A而少采购C)的同时还要满足转运方案的优化目标(减少损耗),
此优化问题的考虑维度增加至二维。为了降低其分析复杂度,我们采用了基于分解的$MOEA/D$法,将其中的切比雪夫算法与与遗传算法结合建立了筛选最优方案的模型。

\subsubsection*{多目标进化算法(MOEA/D)概述}
$MOEA/D$算法于2007年由张清富等人提出,其特点是能够将多目标优化问题分解为多个标量的单目标优化子问题,在较低维度(2$\sim$3)简单PF的多目标优化问题中使用$MOED \ A$算法能在保证结果均匀度与收敛性的同时极大降低计算复杂度,
结合其特点,比较了其他多目标优化方法(如$NSGA-II$)法,我们认为此题更适合使用$MOEVA/D$法分析。

\subsubsection*{使用MOEAD的切比雪夫分解策略}
\begin{description}
    \item[$\blacktriangleright$] \textbf{切比雪夫法} \\
        对于一般的多目标优化问题,通常表示为
        \begin{equation}
            \begin{split}
                maxF(x) = (f_1(x),\dots,f_m(x))^T\nonumber \\
                subject\ to\ x \in \Omega
            \end{split}
        \end{equation}
        即在决策空间中$\Omega$找出一点使目标函数集$F(x)$中的各个目标均取最优值。但由于优化目标之间
        往往存在冲突,不存在使其同时取最优值的解,这时需要在各个解之间权衡出相对最符合要求的解集,即$Pareto$
        最优前沿。\\
        我们使用切比雪夫法分解该问题求解$Pareto$最优前沿,根据其公式
        \begin{equation}
            \begin{split}
                min\ g^{t\varepsilon}(x|\lambda,z*)=\mathop{max}_{{1\le i \le m}}\{\lambda_i|f_i(x)-z_i*|\} \nonumber\\
                subject\ to\ x\in\Omega
            \end{split}
        \end{equation}
        其中参考点$z*=z_1^*\cdots z_m^*$,即$Pareto$最优解集,权重向量为$\lambda$,对于一个分式,决策对象$x$就由这两个量确定。
    \item[$\blacktriangleright$] \textbf{确定参考点} \\
    求解切比雪夫法需要先确定参考点,即分别符合两种优化目标的点。也就是寻找符合题意的运输方案的最优解与订货方案的最优解。
     \begin{itemize}
            \item \textbf{由运输方案最优解得到选购方案$x_1$:}我们沿用了问题2中的近似动态规划法,通过调整A、C两材料的价格权重来获得尽量多采购A而少采购C的选购方案。

            \item \textbf{由订货方案最优解反求选购方案$x_2$:}利用我们提出的贪婪法与0-1背包(动态规划)结合的算法得出最少损耗量对应的转运方案,并以此反求其对应的选购方案。
        \end{itemize}
    \item[$\blacktriangleright$] \textbf{处理权衡向量:}随机生成大量权衡向量用于不同选购方案的评估。
\end{description}
\subsubsection*{建立$MOEA/D$与遗传算法结合的模型}
最终的评判模型结合了$MOEDA/D$与遗传算法,我们对原始数据使用遗传算法遗传生成各种可能的选购方案,并将其用$MOEA/D$法优化,比对结果后留下最优两组结果继续遗传生成、优化、比对$\dots$
重复此步骤直至筛选出最优方案。

\subsubsection{问题三模型的求解}
\begin{figure}[H]
    \centering
    \includegraphics[scale = 0.5]{three.png}
    \centering
    \caption{$MOEA/D$与遗传算法结合程序设计图} \label{fig:6}
\end{figure}
%%%%%%%%%%%程序步骤加框图
\begin{description}

    \item[$\bigstar$] \textbf{问题三求解结果}
\end{description}
\subsection{问题四模型的建立与求解}
\subsubsection{问题四模型的建立}
由于在之前的问题中物流的运货能力由于受到供应方的限制
并未能够被最大程度利用,在供应链管理中,供应方,运输方
,收货方应当建立信息共享机制,
我们在同时考虑到供应商供货能力,运货商运货能力的情况下
重新指定运送方案。
\subsubsection*{求解信息共享后最大化供货能力}
与前一问类似,依然使用贪婪和0-1背包动态规划的方法求得
最大化运货商能力的供货能力,并以此作为提升科技能力后的
产量。
\subsubsection*{求解信息共享后最优转运方案}
由于我们前文已经得出了有供货方案求得转运方案的通用方法,
故此问直接调用程序即可。

\subsubsection{问题四模型的求解}
\begin{description}
    \item[$\blacktriangleright$] \textbf{Step1} 读取转化产能表
    \item[$\blacktriangleright$] \textbf{Step2} 使用贪婪算法对承运商排序
    \item[$\blacktriangleright$] \textbf{Step3} 使用0-1背包动态规划算法
        \begin{itemize}
            \item \textbf{Step3.1} 遍历所有天数
            \item \textbf{Step3.2} 根据产能遍历所有空间
            \item \textbf{Step3.3} 状态转移
        \end{itemize}
    \item[$\blacktriangleright$] \textbf{Step4} 得到转运方案
    \item[$\blacktriangleright$] \textbf{Step5} 反推出最大订货方案
\end{description}



\section{模型的分析与检验}
\subsection{问题一结果的分析与检验}
对于问题一,可以通过对比两种算法的运行结果来检验得到结果的正确性。通过对比两个五十家重要供应商名单的列表,容易发现:结合了熵权法的TOPSIS法与层次分析法得到的结果无论是涵盖的供应商还是其排列顺序都几乎
吻合。两种不同评价方式得出几乎一致的结果,说明二者互相佐证了结论的正确性。\par
此外,我们使用本问确定的四项评价指标对全部402家企业进行了\textbf{聚类分析}
\begin{figure}[H]
    \centering
    \includegraphics[scale = 0.7]{julei.png}
    \centering
    \caption{聚类分析结果,图中圆圈为结果名单中的50家重要企业} \label{fig:5}
\end{figure}
其中竖直方向的坐标轴表示企业供货能力大小,水平方向两条轴表示供货精确度。可以发现,位列50家重要供应商名单中的企业几乎全部被聚类算法划分到了“具有较强供货能力”一类,
而根据本道题其他问的分析与熵权法客观赋权得到的结果都可以说明供货能力是评价供应商对企业重要程度的最主要因素,名单企业几乎均位列此类,充分说明了问题一求解结果的合理性。
\subsection{问题二结果的分析与检验}
以过去5年订货量的平均值作为满足生产需要的最低指标
对问题二使用贪心算法进行求解,得到至少'结果排序'中的前31家企业进行供货能够满足生产需求,但是贪心算法具有短视性,同时按照4项指标进行排序的供货商不一定是该问题的最优解,故使用动态规划进行求解,同时使用近似动态规划解决维度灾问题
由于该算法需要 '开发' 和 '探索' 故得到的结果具有不稳定性,我们取多次求得的均值填入附录A中.
\subsection{问题三结果的分析与检验}
\subsection{问题四结果的分析与检验}
经过对结果进行验证,新方案每周的实际接受量均值相较于原
本的平均供货量提升了$1.68$倍,相较于原来$2.82 \times 10^4$的产能
提高了$1.09$倍
因此,在掌握产能提升的前提下,本企业可以通过技术改革
和合理安排将产能提高至每周$3.1 \times 10^ 4$立方米

\section{模型的评价、改进与推广}
\begin{description}
    \item[$\blacktriangleright$] 第二问中贪心法的使用虽然可以较容易的对平均供货量做出衡量,但是实际考虑制定满足企业生产的方案时不只有平均供货量这一个评判指标,故单纯考虑贪心法具有一定局限性,还需要综合考虑其他计算方法。
    \item[$\blacktriangleright$] 第二问近似动态规划虽然可以避免维度灾,但是由于存在开发和探索两个选项,因此结果存在随机性,并且满足生产需要的阈值由人为评定,具有较大的主观因素 ,运行多次发现结果稳定性并不理想。
下图为三次运行的对应cost结果  

\end{description}

%参考文献
\nocite{*}
\bibliographystyle{plain}
\bibliography{reference}
\newpage
%附录
\begin{appendices}
    \section{支撑材料表}
    \begin{table}[H]
        \begin{center}
            \setlength{\tabcolsep}{0.5mm}{
                \begin{tabular}{ccc}
                    \toprule
                    文件名                                     & 文件类型 & 文件意义                     \\
                    \midrule
                    TOPSIS结果排名                             & Excel    & 使用TOPSIS的排名             \\
                    标准化矩阵和权重                           & Excel    & 如文件名                     \\
                    层次分析表                                 & Excel    & 层次分析的表指标             \\
                    第四题\_理想情况供货                       & Excel    & 信息共享下理想的最大供货表   \\
                    供货商的4项指标评价得分                    & Excel    & 402家供货商4项指标的具体数值 \\
                    通过熵权法TOPSIS得出的综合得分排序\_前50家 & Excel    & 如文件名                     \\
                    在前31家供货商中得出的订购方案             & Excel    & 如文件名                     \\
                    在前50家供货商中得出的订购方案             & Excel    & 如文件名                     \\
                    指标合并权重                               & Excel    & 指标和权重的AHP表            \\
                    转化产能表                                 & Excel    & 将原材料转化为生产量表       \\
                    \bottomrule
                \end{tabular}}
        \end{center}
    \end{table}
    \section{熵权法--matlab程序}
    \subsection{将生产三种原料企业的产能转化为同类型}
    \begin{lstlisting}[language=matlab]
filename = '附件1 近5年402家供应商的相关数据.xlsx';
providesheet = '供应商的供货量(m³)';

[power, text] = xlsread(filename,providesheet);
text = text(2:403,2);

%将三种原料无差异化
for i = 1:402
    divnum = 0;
    if char(text(i)) == 'A'
        divnum = 0.6;
    elseif char(text(i)) == 'B'
        divnum = 0.66;
    else
        divnum =0.72;
    end
    power(i,:) = power(i,:)./divnum;
end

xlswrite('转化产能表.xlsx',power);
\end{lstlisting}
    \subsection{使用熵权法为指标客观赋权}
    \begin{lstlisting}[language=matlab]
filename = '转化产能表.xlsx';
ordersheet = '企业的订货量(m³)';
providesheet = '供应商的供货量(m³)';

%读入数据
order = xlsread(filename,ordersheet);
provide = xlsread(filename,providesheet);

%计算同步率方差和均值
meanratio = [];
varratio = [];
for i = 1:402
  campany = order(i,:);
  provider = provide(i,:);
  ratio = provider./campany;
  numberIndex = find(~isnan(ratio));
  ratio = ratio(numberIndex);
  meanratio = [meanratio, mean(ratio)];
  varratio = [varratio, var(ratio)];
end

%计算总供货量和和单次供货最大值
sumorder = sum(provide, 2);
maxorder = max(provide,[], 2);
xlswrite('指标.xlsx',[meanratio', varratio', sumorder, maxorder]);



%正向化归一化
meanratio = 1 - abs(meanratio - 1)/max(abs(meanratio - 1));
meanratio = unify(meanratio);
meanratio = handleZero(meanratio);

varratio = max(varratio) - varratio;
varratio = unify(varratio);
varratio = handleZero(varratio);

sumorder = unify(sumorder);
sumorder = handleZero(sumorder);

maxorder = unify(maxorder);
maxorder = handleZero(maxorder);

indexMatrix = [meanratio', varratio', sumorder, maxorder];
xlswrite('标准化矩阵和权重.xlsx',indexMatrix,'归一矩阵');

%计算信息熵
for i = 1:4
  line = indexMatrix(:,i);
  line = line ./ sum(line);
  indexMatrix(:,i) = line;
end
E = - (sum(indexMatrix .* log(indexMatrix)))/log(402);

%计算权重
w = (1 - E)/(4 - sum(E));
xlswrite('标准化矩阵和权重.xlsx',w,'权重');

%定义归一化函数
function ret = unify(array)
  ret = (array - min(array))/(max(array) - min(array));
end
function ret = handleZero(array)
  array(find(array == 0)) = 0.00001;
  ret = array;
end
 \end{lstlisting}
    \section{TOPSIS法--matlab程序}
    \begin{lstlisting}[language=matlab]
%W,P 需要导入w(4X1)p(402X4)两个矩阵
clear;

%计算加权规范化矩阵
p = xlsread("标准化矩阵和权重 (1)","归一矩阵");
w = xlsread("标准化矩阵和权重 (1)","权重");
index = xlsread("指标 (1)","sheet1");
Z = zeros(402,4);
for i =1:402
    for j =1:4
        Z(i,j) = p(i,j)*w(j);
    end
end

%通过遍历得到“理想解”与“负理想解”
best =[0,0,0,0];
worst=[1,1,1,1];
[r,c] = size(p);
for j =1:c
    for i =1:r
        if Z(i,j)>best(j)
            best(j) = Z(i,j);
        end
        if Z(i,j)<worst(j)
            worst(j) = Z(i,j);
        end
    end
end

%计算最优、最劣距离
[r,c] = size(Z);
Dplus = zeros(402,1);
Dsub = zeros(402,1);
for i = 1:r
    for j = 1:c
        Dplus(i) = Dplus(i)+ (Z(i,j)-best(j))^2;
        Dsub(i) = Dsub(i)+(Z(i,j)-worst(j))^2;
    end
end
for i =1:r
        Dplus(i) = sqrt(Dplus(i));
        Dsub(i) =sqrt(Dsub(i));      
end

%得到理想贴进度
C = zeros(402,1);
for i = 1:402
    C(i) = Dsub(i)/(Dplus(i) + Dsub(i));
end

%将结果排序并输出前50名
[A,postion] = sort(C);
temp = flipud(postion);
paiming =temp(1:50);
temp = [paiming,index(paiming,:),C(paiming,:)];
xlswrite("结果排名",temp,"sheet1");
\end{lstlisting}
    \section{层次分析法--matlab程序}
    \begin{lstlisting}[language=matlab]
%读入数据
filename = '标准化矩阵和权重.xlsx';
matrixsheet = '归一矩阵';
weightsheet = '权重';

matrix = xlsread(filename,matrixsheet);
weight = xlsread(filename,weightsheet);
zhi = xlsread('指标.xlsx');

%得到加权规范化矩阵
score = weight * matrix';

%计算综合得分并输出前50名
[sortedScore, I] = sort(score);
top50 = flipud(I');
top50 = top50(1:50);
xlswrite('AHP结果.xlsx',[top50, zhi(top50,:),score(top50)']);
\end{lstlisting}
    \section{层次分析法--matlab程序}
    \begin{lstlisting}[language=matlab]
%读入数据
filename = '标准化矩阵和权重.xlsx';
matrixsheet = '归一矩阵';
weightsheet = '权重';

matrix = xlsread(filename,matrixsheet);
weight = xlsread(filename,weightsheet);
zhi = xlsread('指标.xlsx');

%得到加权规范化矩阵
score = weight * matrix';

%计算综合得分并输出前50名
[sortedScore, I] = sort(score);
top50 = flipud(I');
top50 = top50(1:50);
xlswrite('AHP结果.xlsx',[top50, zhi(top50,:),score(top50)']);
\end{lstlisting}



\end{appendices}
\end{document}